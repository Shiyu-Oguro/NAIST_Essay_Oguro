\documentclass[twocolumn]{jarticle}%,2pt
\setlength{\columnsep}{3zw} 

\title{\vspace{5mm}\large{あああああああ}\vspace{-15mm}}
%\title{\tiny{}}
\date{}


\usepackage{textcomp}
\usepackage[sc]{mathpazo}
\usepackage[scaled]{helvet}
\usepackage{amsmath,amssymb}
%\usepackage{otf}%字体が変更できる
\usepackage{color}
\usepackage{threeparttable}
\usepackage[dvipdfmx,hiresbb]{graphicx}%
\usepackage{float}
\usepackage{fancyhdr}
\usepackage{titlesec}
\usepackage[top=30truemm,bottom=30truemm,left=20truemm,right=20truemm]{geometry}

%define for header%\titleformat*{\section}{\large\bfseries}
\titleformat*{\section}{\large\bfseries}
\renewcommand{\ttdefault}

%Header
\pagestyle{fancy}%ページ番号を入れるか決めてるっぽい.
\rhead{}
\lhead{出願区分 : 情報科学区分 \\ 氏名 : 小黒司友 \\ 現在の専門 : 情報工学 \\ 希望希望研究室: インタラクティブメディア設計学研究室}

\begin{document}
\normalsize
%\author %
\maketitle

\section{はじめに}
\thispagestyle{fancy}

私が,奈良先端科学技術大学院大学(以下NAIST)で取り組みたい研究テーマは「ああああああああああ」である.本稿では,\ref{current}章でこれまでの修学内容について,\ref{want}章で貴学で取り組みたい研究テーマの研究背景及び目的・先行研究・提案手法について述べ,結びとする.

\section{現在行っている研究}\label{current}
私が現在行っている研究は,歩行者交通流シミュレータのための歩行モデルの検討である.その詳細を以下に述べる.

\subsection{研究背景}
これまでに,歩行者集団の移動の円滑性・効率性に着目する交通流シミュレータが開発されていた.しかし,歩行者と空間を共有するパーソナルモビリティやロボットを,安全かつ快適に運用するには,個々の歩行者の振る舞いや歩行者間の相互作用までシミュレートする必要がある.

そこで本研究では,歩行者に近い歩行ルール(歩行モデル)を持つアバターを扱うシミュレータを作成することを最終的な目標とする.

%現段階では,シミュレーション環境で各歩行モデルを用いてシミュレーションが行えることを確認した.さらに,アバター操作モードを用いて,被験者がシミュレーションに介入できることを確認した.今後は,歩行モデルとアバター操作モードを用いてシミュレーションを行った結果から,歩行モデルの評価を行う.歩行モデルの評価は,効率性\cite{Iryo-4},安定性\cite{Iryo-4},安全性,みための自然さ\cite{Iryo-2}の観点から行う.

\subsection{シミュレータ}
図\ref{fig:environment}に示すようなシミュレーション環境でシミュレーションを行う.

アバターの歩行開始地点もしくは目的地となる歩道 A,B,C,D を用意する.アバターの流れは様々に設定できるようにする.

空間を2次元平面で考え,アバターは円形領域とする. アバターの流入位置および目的地,円形領域の半径,基本的な歩行速度などは独立に一定の確率分布を与えて決 定する.シミュレーションは離散時間間隔で進行させ,アバターの通し番号や座標,衝突状況などの情報をログファイルに記述する. 

また,被験者がシミュレーション中にどのような行動を行うのかをデータとして得るために,アバターのうちの一体としてアバターを操作し,シミュレーションに参加することができるモード(アバター操作モード) を用いる.被験者は,図\ref{fig:user_view}のような一人称視点で他のアバターの様子を観察しながら操作を行う.

\subsection{歩行モデル}
今回のシミュレータでは,3種類の性質の異なる歩行モデルを用いる.使用するそれぞれの歩行モデルについて,概要を示す.\\

モデル1 : ポテンシャルモデル\cite{Akuzawa}\\
歩行モデルに同一符号の電荷を持たせ,クーロン力により歩行モデル間の斥力を計算し,全てのアバターについて足し合わせる.さらに,目的地に向かう力を加えて最終的にアバターが受ける力を計算する. \\

モデル2 : 追従型モデル\cite{Sasagawa}\\
目的地に向かう基本的な速度 を持ち,前方に他の目的地を持つアバターがいれば回避,同一の目的地のアバターがいれば追従する付加的な速度を加える.\\

モデル3 : 効率重視型モデル\\
アバターに視野を定め,視野内の他のアバターとの衝突を避けつつ目的地に向かう方向と速度を選択する. \\

\begin{figure}[H]
    \begin{tabular}{cc}
      %---- 最初の図 ---------------------------
      \begin{minipage}[t]{0.45\hsize}
        \centering
        \includegraphics[keepaspectratio, scale=0.11]{images/environment.JPG}
        \caption{シミュレーション環境}
        \label{fig:environment}
      \end{minipage} &
      %---- 2番目の図 --------------------------
      \begin{minipage}[t]{0.45\hsize}
        \centering
        \includegraphics[keepaspectratio, scale=0.1]{images/user_view_mini.JPG}
        \caption{被験者が体験するシミュレーションの様子}
        \label{fig:user_view}
      \end{minipage}
      %---- 図はここまで ----------------------
    \end{tabular}
  \end{figure}

\section{貴学において取り組みたい研究}\label{want}
貴学では,

\section{おわりに}

%参考文献
\begin{thebibliography}{1}
\bibitem[1]{Iryo-1} Miho Iryo-Asanoa,Yu Hasegawa,Charitha Dias,
``Applicability of Virtual Reality Systems for Evaluating Pedestrians’ Perception and Behavior''
 (2018)
\bibitem[2]{Iryo-2} Yu HASEGAWA, Miho IRYO-ASANO
``Development of Pedestrian Model for Experiments in Virtual Reality Environment'' (2018)
\bibitem[3]{Iryo-3} Takamasa Iryoa,MihoAsano,Shinta Odani,Shogo Izumi
``Examining factors of walking disutility for microscopic pedestrian model - A virtual reality approach'' 
(2013)
\bibitem[4]{Iryo-4} 井料美帆, 長島愛,
``歩行者交差交通流の性能評価に関する研究''
(2015)
\bibitem[5]{Akuzawa} 阿久澤あずみ
``駅構内における群衆歩行シミュレーションモデルの研究''
\bibitem[6]{Sasagawa} 笹川匠也
``人工現実感を用いた横断歩道における歩行者交通流シミュレータの開発''
\end{thebibliography}

\bibliographystyle{jplain}
\bibliography{Untitled}

\end{document}
