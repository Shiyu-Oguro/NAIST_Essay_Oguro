%参考文献
\begin{thebibliography}{1}
\bibitem[1]{Akuzawa} 阿久澤あずみ
``駅構内における群衆歩行シミュレーションモデルの研究'', 日本オペレーションズリサーチ秋季研究発表会2005,pp192-193, 公益社団法人日本オペレーションズ・リサーチ学会
(2005)
\bibitem[2]{Sasagawa} 笹川匠也
``人工現実感を用いた横断歩道における歩行者交通流シミュレータの開発'', 平成29年度長岡工業高等専門学校卒業論文, 長岡工業高等専門学校電子制御工学科
(2018)
\bibitem[3]{Iryo-Youryou} 井料美帆, 長島愛,
``歩行者交差交通流の性能評価に関する研究'', 生産研究67巻4号, pp369-373
(2015)
\bibitem[4]{Iryo-VRE} Yu HASEGAWA, Miho IRYO-ASANO
``Development of Pedestrian Model for Experiments in Virtual Reality Environment'', 生産研究70巻2号, pp131-135, 東京大学生産技術研究所 (2018)
\bibitem[5]{Iryo-Appli} Miho Iryo-Asanoa,Yu Hasegawa,Charitha Dias,
``Applicability of Virtual Reality Systems for Evaluating Pedestrians’ Perception and Behavior'', Transportation research procedia Volume 34, pp67-74
 (2018)
%\bibitem[6]{Iryo-VRMicroPM} Takamasa Iryo,Miho Asano,Shinta Odani,Shogo Izumi,
%``Examining factors of walking disutility for microscopic pedestrian model - A virtual reality approach'' , Procedia - Social and Behavioral Sciences Volume 80, pp940-959
%(2013)
\end{thebibliography}